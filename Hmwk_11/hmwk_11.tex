\documentclass[]{article}
\usepackage{lmodern}
\usepackage{amssymb,amsmath}
\usepackage{ifxetex,ifluatex}
\usepackage{fixltx2e} % provides \textsubscript
\ifnum 0\ifxetex 1\fi\ifluatex 1\fi=0 % if pdftex
  \usepackage[T1]{fontenc}
  \usepackage[utf8]{inputenc}
\else % if luatex or xelatex
  \ifxetex
    \usepackage{mathspec}
  \else
    \usepackage{fontspec}
  \fi
  \defaultfontfeatures{Ligatures=TeX,Scale=MatchLowercase}
\fi
% use upquote if available, for straight quotes in verbatim environments
\IfFileExists{upquote.sty}{\usepackage{upquote}}{}
% use microtype if available
\IfFileExists{microtype.sty}{%
\usepackage{microtype}
\UseMicrotypeSet[protrusion]{basicmath} % disable protrusion for tt fonts
}{}
\usepackage[margin=1in]{geometry}
\usepackage{hyperref}
\hypersetup{unicode=true,
            pdftitle={Assignment 11},
            pdfauthor={Reid Ginoza},
            pdfborder={0 0 0},
            breaklinks=true}
\urlstyle{same}  % don't use monospace font for urls
\usepackage{color}
\usepackage{fancyvrb}
\newcommand{\VerbBar}{|}
\newcommand{\VERB}{\Verb[commandchars=\\\{\}]}
\DefineVerbatimEnvironment{Highlighting}{Verbatim}{commandchars=\\\{\}}
% Add ',fontsize=\small' for more characters per line
\usepackage{framed}
\definecolor{shadecolor}{RGB}{248,248,248}
\newenvironment{Shaded}{\begin{snugshade}}{\end{snugshade}}
\newcommand{\AlertTok}[1]{\textcolor[rgb]{0.94,0.16,0.16}{#1}}
\newcommand{\AnnotationTok}[1]{\textcolor[rgb]{0.56,0.35,0.01}{\textbf{\textit{#1}}}}
\newcommand{\AttributeTok}[1]{\textcolor[rgb]{0.77,0.63,0.00}{#1}}
\newcommand{\BaseNTok}[1]{\textcolor[rgb]{0.00,0.00,0.81}{#1}}
\newcommand{\BuiltInTok}[1]{#1}
\newcommand{\CharTok}[1]{\textcolor[rgb]{0.31,0.60,0.02}{#1}}
\newcommand{\CommentTok}[1]{\textcolor[rgb]{0.56,0.35,0.01}{\textit{#1}}}
\newcommand{\CommentVarTok}[1]{\textcolor[rgb]{0.56,0.35,0.01}{\textbf{\textit{#1}}}}
\newcommand{\ConstantTok}[1]{\textcolor[rgb]{0.00,0.00,0.00}{#1}}
\newcommand{\ControlFlowTok}[1]{\textcolor[rgb]{0.13,0.29,0.53}{\textbf{#1}}}
\newcommand{\DataTypeTok}[1]{\textcolor[rgb]{0.13,0.29,0.53}{#1}}
\newcommand{\DecValTok}[1]{\textcolor[rgb]{0.00,0.00,0.81}{#1}}
\newcommand{\DocumentationTok}[1]{\textcolor[rgb]{0.56,0.35,0.01}{\textbf{\textit{#1}}}}
\newcommand{\ErrorTok}[1]{\textcolor[rgb]{0.64,0.00,0.00}{\textbf{#1}}}
\newcommand{\ExtensionTok}[1]{#1}
\newcommand{\FloatTok}[1]{\textcolor[rgb]{0.00,0.00,0.81}{#1}}
\newcommand{\FunctionTok}[1]{\textcolor[rgb]{0.00,0.00,0.00}{#1}}
\newcommand{\ImportTok}[1]{#1}
\newcommand{\InformationTok}[1]{\textcolor[rgb]{0.56,0.35,0.01}{\textbf{\textit{#1}}}}
\newcommand{\KeywordTok}[1]{\textcolor[rgb]{0.13,0.29,0.53}{\textbf{#1}}}
\newcommand{\NormalTok}[1]{#1}
\newcommand{\OperatorTok}[1]{\textcolor[rgb]{0.81,0.36,0.00}{\textbf{#1}}}
\newcommand{\OtherTok}[1]{\textcolor[rgb]{0.56,0.35,0.01}{#1}}
\newcommand{\PreprocessorTok}[1]{\textcolor[rgb]{0.56,0.35,0.01}{\textit{#1}}}
\newcommand{\RegionMarkerTok}[1]{#1}
\newcommand{\SpecialCharTok}[1]{\textcolor[rgb]{0.00,0.00,0.00}{#1}}
\newcommand{\SpecialStringTok}[1]{\textcolor[rgb]{0.31,0.60,0.02}{#1}}
\newcommand{\StringTok}[1]{\textcolor[rgb]{0.31,0.60,0.02}{#1}}
\newcommand{\VariableTok}[1]{\textcolor[rgb]{0.00,0.00,0.00}{#1}}
\newcommand{\VerbatimStringTok}[1]{\textcolor[rgb]{0.31,0.60,0.02}{#1}}
\newcommand{\WarningTok}[1]{\textcolor[rgb]{0.56,0.35,0.01}{\textbf{\textit{#1}}}}
\usepackage{graphicx,grffile}
\makeatletter
\def\maxwidth{\ifdim\Gin@nat@width>\linewidth\linewidth\else\Gin@nat@width\fi}
\def\maxheight{\ifdim\Gin@nat@height>\textheight\textheight\else\Gin@nat@height\fi}
\makeatother
% Scale images if necessary, so that they will not overflow the page
% margins by default, and it is still possible to overwrite the defaults
% using explicit options in \includegraphics[width, height, ...]{}
\setkeys{Gin}{width=\maxwidth,height=\maxheight,keepaspectratio}
\IfFileExists{parskip.sty}{%
\usepackage{parskip}
}{% else
\setlength{\parindent}{0pt}
\setlength{\parskip}{6pt plus 2pt minus 1pt}
}
\setlength{\emergencystretch}{3em}  % prevent overfull lines
\providecommand{\tightlist}{%
  \setlength{\itemsep}{0pt}\setlength{\parskip}{0pt}}
\setcounter{secnumdepth}{0}
% Redefines (sub)paragraphs to behave more like sections
\ifx\paragraph\undefined\else
\let\oldparagraph\paragraph
\renewcommand{\paragraph}[1]{\oldparagraph{#1}\mbox{}}
\fi
\ifx\subparagraph\undefined\else
\let\oldsubparagraph\subparagraph
\renewcommand{\subparagraph}[1]{\oldsubparagraph{#1}\mbox{}}
\fi

%%% Use protect on footnotes to avoid problems with footnotes in titles
\let\rmarkdownfootnote\footnote%
\def\footnote{\protect\rmarkdownfootnote}

%%% Change title format to be more compact
\usepackage{titling}

% Create subtitle command for use in maketitle
\providecommand{\subtitle}[1]{
  \posttitle{
    \begin{center}\large#1\end{center}
    }
}

\setlength{\droptitle}{-2em}

  \title{Assignment 11}
    \pretitle{\vspace{\droptitle}\centering\huge}
  \posttitle{\par}
    \author{Reid Ginoza}
    \preauthor{\centering\large\emph}
  \postauthor{\par}
      \predate{\centering\large\emph}
  \postdate{\par}
    \date{4/26/2020}

\newcommand{\D}{\mathrm{d}}
\DeclareMathOperator*{\argmin}{arg\,min}

\begin{document}
\maketitle

\hypertarget{scalar-sde-analytical-solution}{%
\section{Scalar SDE Analytical
Solution}\label{scalar-sde-analytical-solution}}

This is Exercise 8.1 from (Särkkä and Solin 2019). We are given the
following SDE: \begin{equation}
\mathrm{d}x = -c \, x \mathrm{d}t + g \, x \mathrm{d}\beta, \quad x(0) = x_0,
\end{equation} where \(c\), \(g\), and \(x_0\) are positive constants
and \(\beta (t)\) is a standard Brownian motion.

We will use the Itô chain rule, where the drift function is
\(F(x) = -cx\) and the dispersion function \(G(x) = gx\). We want to
solve for \(u(x) = \ln{\left( x\right)}\). \begin{align}
u_t &= 0\\
u_x &= \dfrac{1}{x}\\
u_{xx} &= - \dfrac{1}{x^2}\\
\implies \mathrm{d}u &= \left(u_t + u_x F + \dfrac{1}{2} u_{xx} G^2 \right) \mathrm{d}t + u_x \,G \,\mathrm{d}\beta\\
&=\left(0 - \dfrac{cx}{x} - \dfrac{g^2 x^2}{2x^2} \right) \mathrm{d}t + \dfrac{gx}{x} \, \mathrm{d}\beta\\
&=\left(- c - \dfrac{g^2}{2} \right) \mathrm{d}t + g \, \mathrm{d}\beta\\
\implies u &= \left(- c - \dfrac{g^2}{2} \right) t + g \beta + C\\
\ln{\left( x \right)} &= \left(- c - \dfrac{g^2}{2} \right) t + g \beta + C\\
\implies x &= C \exp{\left[ \left(- c - \dfrac{g^2}{2} \right) t + g \beta \right]}\\
\text{Applying the initial condition,}\quad\quad x &= x_0 \exp{\left[ \left(- c - \dfrac{g^2}{2} \right) t + g \beta \right]}
\end{align}

\hypertarget{numerical-solution}{%
\section{Numerical Solution}\label{numerical-solution}}

We will use the Milstein method for the an SDE in this form:
\begin{equation}
\mathrm{d}x = F(x) \mathrm{d}t + G(x) \mathrm{d}\beta,
\end{equation} where \(x\) is a scalar, \(\beta\) is the standard
Brownian motion, and we given the initial condition \(x(0) = x_0\).
Given the Brownian motion step \(\Delta\beta\), or having drawn it from
\(\Delta \beta \sim \mathrm{N}\left(0, \Delta t\right)\), the Milstein
calculates each time step \(\hat{x}\) as: \begin{equation}
\hat{x} (t_{k+1}) = \hat{x} (t_k) + F(\hat{x}(t_k)) \Delta t 
+ G(\hat{x}(t_k)) \Delta \beta_k
+ \dfrac{1}{2} \dfrac{\partial\,G (\hat{x}(t_k))}{\partial x} G (\hat{x}(t_k)) (\Delta \beta_k^2 - \Delta t)
\end{equation}

We will examine the case where \(x_0 = 1, \, c = 0.1, \, g=0.1\) with
the Milstein method. The SDE, therefore, is \begin{equation}
\mathrm{d}x = -0.1x \mathrm{d}t + 0.1 x \mathrm{d}\beta, \quad x(0) = 1.
\end{equation} and the known analytical solution is: \begin{equation}
x = \exp{\left[ -0.105 t + 0.1 \beta\right]}
\end{equation} and the Milstein step, given the Brownian motion step
\(\Delta \beta_k\) is: \begin{equation}
\hat{x} (t_{k+1}) = \hat{x} (t_k) - 0.1 \hat{x}(t_k) \Delta t 
+ 0.1 \hat{x}(t_k) \Delta \beta_k
+ 0.005 \hat{x}(t_k) (\Delta \beta_k^2 - \Delta t)
\end{equation}

\begin{Shaded}
\begin{Highlighting}[]
\ImportTok{import}\NormalTok{ numpy }\ImportTok{as}\NormalTok{ np}
\ImportTok{from}\NormalTok{ matplotlib }\ImportTok{import}\NormalTok{ pyplot }\ImportTok{as}\NormalTok{ plt}

\CommentTok{# -- Brownian Motion --}
\KeywordTok{def}\NormalTok{ multiple_brownian_motion(end_time}\OperatorTok{=}\FloatTok{1.}\NormalTok{, num_tsteps}\OperatorTok{=}\DecValTok{500}\NormalTok{, n_trials}\OperatorTok{=}\DecValTok{1000}\NormalTok{):}
    \CommentTok{"""Creates multiple 1-D Brownian motion with time as the row index and}
\CommentTok{    each column as a separate path of Brownian motion.}

\CommentTok{    This assumes that all Brownian motion starts at 0. Currently only}
\CommentTok{    implements one-dimensional Brownian motion. This also assumes all}
\CommentTok{    step sizes are the same size.}

\CommentTok{    The steps of Brownian motion, ``dw``, are modeled with a Gaussian}
\CommentTok{    distribution with mean 0 and variance ``sqrt(dt)``, where ``dt``}
\CommentTok{    is the constant time step size.}

\CommentTok{    Parameters}
\CommentTok{    ----------}
\CommentTok{    end_time : float}

\CommentTok{    num_tsteps : int}
\CommentTok{        The number of steps to take. Will calculate the step}
\CommentTok{        size dt internally. The number of rows in the output of}
\CommentTok{        Brownian motion will be num_tsteps + 1.}

\CommentTok{    n_trials : int}
\CommentTok{        The number of sample paths to create. This will be the number}
\CommentTok{        of columns in the output.}

\CommentTok{    Returns}
\CommentTok{    -------}
\CommentTok{    t : ndarray}
\CommentTok{        One-dimensional time ndarray from 0 to ``end_time`` with}
\CommentTok{        shape (``num_tsteps``+1,)}

\CommentTok{    w : ndarray}
\CommentTok{        Two-dimensional ndarray representing ``n_trials`` number of}
\CommentTok{        sample paths of one-dimensional Brownian motion.}
\CommentTok{        This will be of shape (``num_tsteps``+1, ``n_trials``).}

\CommentTok{    dt : float}
\CommentTok{        The value indicating the step size of t. This is only implemented}
\CommentTok{        with constant step size.}

\CommentTok{    dw : ndarray}
\CommentTok{        Two-dimensional ndarray representing the steps of Brownian motion.}
\CommentTok{        The first row is all zeros. Each i-th row of ``dw``, ie. dw[i, :]}
\CommentTok{        indicates the change in ``w`` from w[i-1, :] to w[i, :]}
\CommentTok{        This will be the same shape as ``w``, (``num_tsteps``+1, ``n_trials``).}

\CommentTok{    """}

\NormalTok{    dt }\OperatorTok{=}\NormalTok{ (end_time }\OperatorTok{-} \DecValTok{0}\NormalTok{) }\OperatorTok{/}\NormalTok{ num_tsteps}
\NormalTok{    dw }\OperatorTok{=}\NormalTok{ np.random.normal(scale}\OperatorTok{=}\NormalTok{np.sqrt(dt), size}\OperatorTok{=}\NormalTok{(num_tsteps}\OperatorTok{+}\DecValTok{1}\NormalTok{, n_trials))}
    \CommentTok{# Brownian motion must start at time 0 with value 0}
\NormalTok{    dw[}\DecValTok{0}\NormalTok{] }\OperatorTok{=}\NormalTok{ np.zeros_like(dw[}\DecValTok{0}\NormalTok{])}
\NormalTok{    w }\OperatorTok{=}\NormalTok{ dw.cumsum(axis}\OperatorTok{=}\DecValTok{0}\NormalTok{)}
    \CommentTok{# t is not used in calculations, but returned to allow user to keep track}
    \CommentTok{# of points in time}
\NormalTok{    t }\OperatorTok{=}\NormalTok{ np.linspace(}\DecValTok{0}\NormalTok{, end_time, num}\OperatorTok{=}\NormalTok{num_tsteps}\OperatorTok{+}\DecValTok{1}\NormalTok{).reshape((num_tsteps}\OperatorTok{+}\DecValTok{1}\NormalTok{, }\DecValTok{1}\NormalTok{))}
    \ControlFlowTok{assert}\NormalTok{ w.shape[}\DecValTok{0}\NormalTok{] }\OperatorTok{==}\NormalTok{ t.shape[}\DecValTok{0}\NormalTok{], (}\StringTok{'time and position arrays are not the '}
                                      \StringTok{'same length. w.shape[0] - t.shape[0] = '}
                                      \SpecialStringTok{f'}\SpecialCharTok{\{w.}\NormalTok{shape[}\DecValTok{0}\NormalTok{] }\OperatorTok{-} \SpecialCharTok{t.}\NormalTok{shape[}\DecValTok{0}\NormalTok{]}\SpecialCharTok{\}}\SpecialStringTok{'}\NormalTok{)}
    \ControlFlowTok{assert}\NormalTok{ w.shape }\OperatorTok{==}\NormalTok{ dw.shape, (}\StringTok{'position and velocity arrays are not the '}
                                 \StringTok{'same shape: '}
                                 \SpecialStringTok{f'w.shape: }\SpecialCharTok{\{w.}\NormalTok{shape}\SpecialCharTok{\}}\SpecialStringTok{    dw.shape: }\SpecialCharTok{\{dw.}\NormalTok{shape}\SpecialCharTok{\}}\SpecialStringTok{'}\NormalTok{)}
    \ControlFlowTok{return}\NormalTok{ t, w, dt, dw}


\KeywordTok{def}\NormalTok{ euler_maruyama_nonlinear_vec(f, g, x0, t, dt, dw, M):}
    \CommentTok{"""}
\CommentTok{    calculates the EM approximation on the nonlinear one-dimensional SDE,}
\CommentTok{    vectorized for multiple trials based on the shape of ``dw``.}

\CommentTok{    SDE is of the form:}
\CommentTok{    dX = f(X)dt + G(X)dW; X(0) = X0}

\CommentTok{    :param f: shift function in SDE. Must be passed as a function of x}
\CommentTok{    :param g: drift/dispersion function in SDE}
\CommentTok{    :param x0: Initial condition}
\CommentTok{    :param t: time one dimensional nd-array}
\CommentTok{    :param dt: step size of time array, float}
\CommentTok{    :param dw: White noise associated with the Brownian motion, ndarray}
\CommentTok{    :param M: multiple of dt for Euler-Maruyama step size. Do not make this too large.}
\CommentTok{    :return: time array and solution x array}
\CommentTok{    """}

    \ControlFlowTok{if}\NormalTok{ M }\OperatorTok{<} \DecValTok{1}\NormalTok{:}
        \ControlFlowTok{raise} \PreprocessorTok{ValueError}\NormalTok{(}\StringTok{'M must be greater than or equal to 1'}\NormalTok{)}

\NormalTok{    Dt }\OperatorTok{=}\NormalTok{ M }\OperatorTok{*}\NormalTok{ dt  }\CommentTok{# EM step size}
\NormalTok{    L }\OperatorTok{=}\NormalTok{ (t.shape[}\DecValTok{0}\NormalTok{] }\OperatorTok{-} \DecValTok{1}\NormalTok{) }\OperatorTok{/}\NormalTok{ M  }\CommentTok{# number of EM steps}

    \ControlFlowTok{if} \KeywordTok{not}\NormalTok{ L.is_integer():}
        \ControlFlowTok{raise} \PreprocessorTok{ValueError}\NormalTok{(}\StringTok{'Cannot handle Step Size that is not a multiple of M'}\NormalTok{)}

\NormalTok{    L }\OperatorTok{=} \BuiltInTok{int}\NormalTok{(L)  }\CommentTok{# needed for range below}

\NormalTok{    x }\OperatorTok{=}\NormalTok{ [np.full((dw.shape[}\DecValTok{1}\NormalTok{],), x0)]}
\NormalTok{    T }\OperatorTok{=}\NormalTok{ [}\DecValTok{0}\NormalTok{]}
    \ControlFlowTok{for}\NormalTok{ i }\KeywordTok{in} \BuiltInTok{range}\NormalTok{(}\DecValTok{1}\NormalTok{, L}\OperatorTok{+}\DecValTok{1}\NormalTok{):}
        \CommentTok{# DW is the step of Brownian motion for EM step size}
\NormalTok{        DW }\OperatorTok{=}\NormalTok{ (dw[M }\OperatorTok{*}\NormalTok{ (i }\OperatorTok{-} \DecValTok{1}\NormalTok{) }\OperatorTok{+} \DecValTok{1}\NormalTok{:M }\OperatorTok{*}\NormalTok{ i }\OperatorTok{+} \DecValTok{1}\NormalTok{, :]).}\BuiltInTok{sum}\NormalTok{(axis}\OperatorTok{=}\DecValTok{0}\NormalTok{).reshape(dw.shape[}\DecValTok{1}\NormalTok{], )}
\NormalTok{        x.append(x[i}\DecValTok{-1}\NormalTok{] }\OperatorTok{+}\NormalTok{ Dt }\OperatorTok{*}\NormalTok{ f(x[i}\DecValTok{-1}\NormalTok{]) }\OperatorTok{+}\NormalTok{ g(x[i}\DecValTok{-1}\NormalTok{]) }\OperatorTok{*}\NormalTok{ DW)}
\NormalTok{        T.append(T[i}\DecValTok{-1}\NormalTok{] }\OperatorTok{+}\NormalTok{ Dt)}

    \ControlFlowTok{return}\NormalTok{ np.array(T), np.array(x)}


\KeywordTok{def}\NormalTok{ milstein(f, g, dg, x0, t, dt, dw, M):}
    \ControlFlowTok{if}\NormalTok{ M }\OperatorTok{<} \DecValTok{1}\NormalTok{:}
        \ControlFlowTok{raise} \PreprocessorTok{ValueError}\NormalTok{(}\StringTok{'M must be greater than or equal to 1'}\NormalTok{)}

\NormalTok{    Dt }\OperatorTok{=}\NormalTok{ M }\OperatorTok{*}\NormalTok{ dt  }\CommentTok{# EM step size}
\NormalTok{    L }\OperatorTok{=}\NormalTok{ (t.shape[}\DecValTok{0}\NormalTok{] }\OperatorTok{-} \DecValTok{1}\NormalTok{) }\OperatorTok{/}\NormalTok{ M  }\CommentTok{# number of EM steps}

    \ControlFlowTok{if} \KeywordTok{not}\NormalTok{ L.is_integer():}
        \ControlFlowTok{raise} \PreprocessorTok{ValueError}\NormalTok{(}\StringTok{'Cannot handle Step Size that is not a multiple of M'}\NormalTok{)}

\NormalTok{    L }\OperatorTok{=} \BuiltInTok{int}\NormalTok{(L)  }\CommentTok{# needed for range below}

\NormalTok{    x }\OperatorTok{=}\NormalTok{ [np.full((dw.shape[}\DecValTok{1}\NormalTok{],), x0)]}
\NormalTok{    T }\OperatorTok{=}\NormalTok{ [}\DecValTok{0}\NormalTok{]}
    \ControlFlowTok{for}\NormalTok{ i }\KeywordTok{in} \BuiltInTok{range}\NormalTok{(}\DecValTok{1}\NormalTok{, L}\OperatorTok{+}\DecValTok{1}\NormalTok{):}
        \CommentTok{# DW is the step of Brownian motion for EM step size}
\NormalTok{        DW }\OperatorTok{=}\NormalTok{ (dw[M }\OperatorTok{*}\NormalTok{ (i }\OperatorTok{-} \DecValTok{1}\NormalTok{) }\OperatorTok{+} \DecValTok{1}\NormalTok{:M }\OperatorTok{*}\NormalTok{ i }\OperatorTok{+} \DecValTok{1}\NormalTok{, :]).}\BuiltInTok{sum}\NormalTok{(axis}\OperatorTok{=}\DecValTok{0}\NormalTok{).reshape(dw.shape[}\DecValTok{1}\NormalTok{], )}
\NormalTok{        x.append(x[i}\DecValTok{-1}\NormalTok{] }\OperatorTok{+}\NormalTok{ Dt }\OperatorTok{*}\NormalTok{ f(x[i}\DecValTok{-1}\NormalTok{]) }\OperatorTok{+}\NormalTok{ g(x[i}\DecValTok{-1}\NormalTok{]) }\OperatorTok{*}\NormalTok{ DW}
                 \OperatorTok{+} \FloatTok{0.5} \OperatorTok{*}\NormalTok{ dg(x[i}\DecValTok{-1}\NormalTok{]) }\OperatorTok{*}\NormalTok{ g(x[i}\DecValTok{-1}\NormalTok{]) }\OperatorTok{*}\NormalTok{ (DW}\OperatorTok{**}\DecValTok{2} \OperatorTok{-}\NormalTok{ dt))}
\NormalTok{        T.append(T[i}\DecValTok{-1}\NormalTok{] }\OperatorTok{+}\NormalTok{ Dt)}

    \ControlFlowTok{return}\NormalTok{ np.array(T), np.array(x)}


\CommentTok{# -- Input Deck --}
\KeywordTok{def}\NormalTok{ f(x):}
    \ControlFlowTok{return} \OperatorTok{-} \FloatTok{0.1} \OperatorTok{*}\NormalTok{ x}


\KeywordTok{def}\NormalTok{ g(x):}
    \ControlFlowTok{return} \FloatTok{0.1} \OperatorTok{*}\NormalTok{ x}


\KeywordTok{def}\NormalTok{ dg(x):}
    \KeywordTok{del}\NormalTok{ x  }\CommentTok{# unused}
    \ControlFlowTok{return} \FloatTok{0.1}


\CommentTok{# Initial Condition}
\NormalTok{x0 }\OperatorTok{=} \DecValTok{1}

\CommentTok{# Known Analytical Solution}
\KeywordTok{def}\NormalTok{ x_true(t, w):}
    \ControlFlowTok{return}\NormalTok{ np.exp(}\OperatorTok{-}\FloatTok{0.105}\OperatorTok{*}\NormalTok{t }\OperatorTok{+} \FloatTok{0.1}\OperatorTok{*}\NormalTok{w)}


\CommentTok{# simulation variables}
\NormalTok{END_TIME }\OperatorTok{=} \DecValTok{1}
\NormalTok{NUM_TSTEPS }\OperatorTok{=} \DecValTok{500}
\NormalTok{N_TRIALS }\OperatorTok{=} \DecValTok{1000}
\NormalTok{APPROX_STEP }\OperatorTok{=} \DecValTok{50}

\CommentTok{# plotting}
\NormalTok{color_rotate }\OperatorTok{=} \DecValTok{10}
\NormalTok{color_num }\OperatorTok{=} \BuiltInTok{int}\NormalTok{(N_TRIALS}\OperatorTok{/}\NormalTok{color_rotate)}
\NormalTok{colors }\OperatorTok{=}\NormalTok{ plt.get_cmap(}\StringTok{'viridis'}\NormalTok{, lut}\OperatorTok{=}\NormalTok{color_num)}
\end{Highlighting}
\end{Shaded}

\hypertarget{sample-paths}{%
\subsection{Sample Paths}\label{sample-paths}}

This section shows sample paths generated by the true solution given
1000 samples of brownian motion, and the approximations from the Euler
Maruyama solution and Milstein solution, both using only 0.2 times the
number of steps as the true solution.

\begin{Shaded}
\begin{Highlighting}[]
\CommentTok{# start run}
\NormalTok{t, w, dt, dw }\OperatorTok{=}\NormalTok{ multiple_brownian_motion(END_TIME, NUM_TSTEPS, N_TRIALS)}
\NormalTok{full_true }\OperatorTok{=}\NormalTok{ x_true(t, w)}

\NormalTok{t_em, x_em }\OperatorTok{=}\NormalTok{ euler_maruyama_nonlinear_vec(f, g, x0, t, dt, dw, M}\OperatorTok{=}\NormalTok{APPROX_STEP)}
\NormalTok{t_mil, x_mil }\OperatorTok{=}\NormalTok{ milstein(f, g, dg, x0, t, dt, dw, M}\OperatorTok{=}\NormalTok{APPROX_STEP)}

\CommentTok{# plot true solution}

\NormalTok{fig_path, ax_path }\OperatorTok{=}\NormalTok{ plt.subplots(nrows}\OperatorTok{=}\DecValTok{3}\NormalTok{, sharex}\OperatorTok{=}\VariableTok{True}\NormalTok{, sharey}\OperatorTok{=}\VariableTok{True}\NormalTok{, figsize}\OperatorTok{=}\NormalTok{(}\DecValTok{8}\NormalTok{, }\DecValTok{10}\NormalTok{))}
\ControlFlowTok{for}\NormalTok{ idx, trial }\KeywordTok{in} \BuiltInTok{enumerate}\NormalTok{(full_true.transpose()):}
\NormalTok{    ax_path[}\DecValTok{0}\NormalTok{].plot(t, trial, color}\OperatorTok{=}\NormalTok{colors(idx}\OperatorTok\NormalTok{color_num), alpha}\OperatorTok{=}\FloatTok{0.8}\NormalTok{)}
\NormalTok{ax_path[}\DecValTok{1}\NormalTok{].set_ylabel(}\StringTok{'Value'}\NormalTok{)}
\NormalTok{ax_path[}\DecValTok{1}\NormalTok{].set_title(}\StringTok{'Euler-Maruyama Solution'}\NormalTok{)}

\ControlFlowTok{for}\NormalTok{ idx, trial }\KeywordTok{in} \BuiltInTok{enumerate}\NormalTok{(x_mil.transpose()):}
\NormalTok{    ax_path[}\DecValTok{2}\NormalTok{].plot(t_mil, trial, color}\OperatorTok{=}\NormalTok{colors(idx}\OperatorTok{%}\NormalTok{color_num), alpha}\OperatorTok{=}\FloatTok{0.8}\NormalTok{)}
\NormalTok{ax_path[}\DecValTok{2}\NormalTok{].set_title(}\StringTok{'Milstein Solution'}\NormalTok{)}
\NormalTok{ax_path[}\DecValTok{2}\NormalTok{].set_xlabel(}\StringTok{'Time'}\NormalTok{)}
\CommentTok{# fig_path.subplots_adjust(top=0.88)}
\CommentTok{# fig_path.suptitle('Paths')}
\NormalTok{plt.tight_layout()}
\NormalTok{plt.show()}
\end{Highlighting}
\end{Shaded}

\includegraphics{/Users/macowner/Documents/Stochastic_Calculus/Hmwk_11/hmwk_11_files/figure-latex/sample paths-1.pdf}

\hypertarget{distribution-at-the-end-time}{%
\subsection{Distribution at the End
Time}\label{distribution-at-the-end-time}}

Now we show the distributions at the end time \(t=1\). Surprisingly,
even with the large time steps, both the Euler-Maruyama and the Milstein
method perform well in matching the distribution.

\begin{Shaded}
\begin{Highlighting}[]
\CommentTok{# Look at end_time}
\NormalTok{fig, ax }\OperatorTok{=}\NormalTok{ plt.subplots(nrows}\OperatorTok{=}\DecValTok{3}\NormalTok{, sharex}\OperatorTok{=}\VariableTok{True}\NormalTok{, sharey}\OperatorTok{=}\VariableTok{True}\NormalTok{)}
\NormalTok{h0 }\OperatorTok{=}\NormalTok{ ax[}\DecValTok{0}\NormalTok{].hist(full_true[}\OperatorTok{-}\DecValTok{1}\NormalTok{:].flatten(), bins}\OperatorTok{=}\StringTok{'auto'}\NormalTok{)}
\NormalTok{ax[}\DecValTok{0}\NormalTok{].set_title(}\StringTok{'True Solution'}\NormalTok{)}
\NormalTok{h1 }\OperatorTok{=}\NormalTok{ ax[}\DecValTok{1}\NormalTok{].hist(x_em[}\OperatorTok{-}\DecValTok{1}\NormalTok{:].flatten(), bins}\OperatorTok{=}\StringTok{'auto'}\NormalTok{)}
\NormalTok{ax[}\DecValTok{1}\NormalTok{].set_title(}\StringTok{'Euler-Maruyama Solution'}\NormalTok{)}
\NormalTok{h2 }\OperatorTok{=}\NormalTok{ ax[}\DecValTok{2}\NormalTok{].hist(x_mil[}\OperatorTok{-}\DecValTok{1}\NormalTok{:].flatten(), bins}\OperatorTok{=}\StringTok{'auto'}\NormalTok{)}
\NormalTok{ax[}\DecValTok{2}\NormalTok{].set_title(}\StringTok{'Milstein Solution'}\NormalTok{)}
\CommentTok{# fig.subplots_adjust(top=0.88)}
\CommentTok{# fig.suptitle(f'Distribution at Time $t=\{}\RegionMarkerTok{END}\CommentTok{_TIME\}$')}
\NormalTok{plt.tight_layout()}
\NormalTok{plt.show()}
\end{Highlighting}
\end{Shaded}

\includegraphics{/Users/macowner/Documents/Stochastic_Calculus/Hmwk_11/hmwk_11_files/figure-latex/end time distribution-1.pdf}

\hypertarget{references}{%
\section*{References}\label{references}}
\addcontentsline{toc}{section}{References}

\hypertarget{refs}{}
\leavevmode\hypertarget{ref-sarkka2019applied}{}%
Särkkä, Simo, and Arno Solin. 2019. \emph{Applied Stochastic
Differential Equations}. Vol. 10. Cambridge University Press.


\end{document}
